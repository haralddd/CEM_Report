\documentclass[../main.tex]{subfiles}
\graphicspath{{\subfix{../Images/}}}

\begin{document}
\section{Numerical method}
The numerical approach of solving the reduced Rayleigh 
equation consists of discretizing the coordinates in both $x_1$
and its reciprocal space, spanned by $(p,q)$.
Additionally, approximative measures are taken to reduce the computational loads 
from calculating an infinite number of modes in \autoref{eq:RRE} and 
the integral function \autoref{I_integral}.
After this is done, 
the Reduced Rayleigh equation can simply be solved as a linear system of equations.


\subsection{Discretization}
Since the $q$-integral in \autoref{eq:RRE} spans an infinite range, 
we consider that the wavenumber values above $\omega/c$ 
represent rapidly decaying evanescent waves, 
since $\alpha_0$ turns purely imaginary.
Based on this we choose to truncate the integral bounds at $Q/2$,
\begin{equation}
    \mathlarger{\mathlarger{\int}}^{\frac{Q}{2}}_{-\frac{Q}{2}} \frac{dq}{2\pi} M_\nu^+(p|q)R(q|k) = M_\nu^-(p|k),
\end{equation}
where $Q$ is a sufficiently large multiple of $2\omega/c$, 
dependent on how many significant evanescent modes is needed for accurate results. 
The discretization of $q$ and $p$ is then defined by 
\begin{alignat*}{2}
    q \longrightarrow&\ q_n = nh_q - Q/2,\quad &n = 0,1,...,N_q\\
    p \longrightarrow&\ p_m = mh_q - Q/2,\quad &m = 0,1,...,N_q
\end{alignat*}
where $h_q = Q/N_q$, such that $-Q \leq p_m - q_n \leq Q$. 
The truncated integral expression can be discretized as a quadrature sum
\begin{equation} \label{eq:discrete_RRE}
    \frac{h_q}{2\pi} \sum^{N_q}_{n=0}
    M_\nu^+(p_m|q_n)R(q_n|k) = M_\nu^-(p_m|k),
\end{equation}
where we have taken all quadrature weights equal to one. 
This defines a system of $N_q$ linear equations, 
but it remains to discretize the surface coordinates 
and handle the integral $I(\gamma|q)$ defined in \autoref{I_integral}. 
Discretizing this integral term results in a discrete Fourier transform at each point $p_m$. 
To avoid calculating $N_q$ Fourier transformations at each surface realization, 
we rewrite the exponential as an infinite sum
\begin{equation*}
    I(\gamma|\Tilde{q}) = \int_{-\infty}^{\infty}dx_1 
        e^{i\gamma\zeta(x_1)} e^{-i\Tilde{q}x_1}
            = \int_{-\infty}^{\infty}dx_1 
            e^{-i\Tilde{q}x_1}
            \sum_{j=0}^{\infty}
                \frac{ \left[i\gamma\zeta(x_1) \right]^j}{ j! },
\end{equation*}
and since the sum is convergent, 
we can exchange the integral and summation, 
making the expression equivalent to
\begin{equation}\label{eq:I_expanded}
    I(\gamma | \Tilde{q}) = \sum_{j=0}^\infty \frac{ \left[i\gamma\right]^j}{j!}\mathcal{F}\left\{\zeta^j(x_1)\right\}
\end{equation}

where $\mathcal{F}$ denotes the Fourier transform from $x_1$ to $\Tilde{q}$. 
The sum is then truncated at an upper limit $N_j$ 
to reduce the number of required Fourier transforms per surface realization. 
The truncation error of this approximation 
is highly dependent on the roughness of the surface.

The integral and its parameters take the form $I(\alpha(p,\omega) - \alpha_0(q,\omega) | p_m - q_n)$, 
imposing a requirement on the discretization of $x_1$ due to the Fourier transform in \autoref{eq:I_expanded}. 
The discrete Fourier transform must be able to resolve $-Q \leq p_m - q_n \leq Q$ for all $m$ and $n$. 
This indicates the requirement $N_x \geq 2N_q$, where we take $N_x = 2N_q$ for simplicity. 
Truncating the surface length at $L$ then gives the discretization
\begin{equation}
    x_1 \longrightarrow x_\ell = \ell h_x - L/2,\quad \ell = 0,1,...,2N_q,
\end{equation}
where $h_x = L/2N_q$. 
The truncation of the surface length results in a discrete Fourier transform
which represents a periodically repeating surface.
From this it is possible to observe boundary effects 
which do not represent the properties of the generated surface.
This is comes into play on especially rough surfaces,
as an increase in propagating surface waves can couple to an anomalous surface effects.

% \begin{equation}
%     I(\gamma|\hat{Q}) = \int_{-\infty}^{\infty} dx_1 e^{i\gamma\zeta(x_1)} e^{-i\hat{Q}x_1}; \; \hat{Q} \in \{-Q, -Q+h_q, .\: .\: .\: , Q-h_q,  Q \}
% \end{equation}


\subsection{Surface sample generation}
A straightforward method for generating randomly rough surfaces is Fourier filtering, 
which consist of drawing a random sample and filtering
it with the surface autocorrelation function.
This is a very convenient approach, 
especially when considering Gaussian rough surfaces, 
as the central limit theorem does not affect the wanted characteristics of the surface.

The generation of $N_x$ surface points $\zeta_\ell$ can then be created by
\begin{equation}
    \zeta_\ell = \mathcal{F}^{-1} \left(\mathcal{F} S_\ell \cdot \sqrt{\mathcal{F} W(x_\ell)}\right),\quad n = 0,1,...,N_x
\end{equation}
where $S_\ell$ is an independent sample from the standard normal distribution, 
$\mathcal{N}(0,1)$, and $\mathcal{F}$ is the discrete Fourier transform.
The autocorrelation function $W(|x|)$ is defined in \autoref{eq:gauss_corr}.

\subsection{Numerical reflection coefficient}


\subsection{Normalization}

\subsection{Solving the reduced Rayleigh equation}

To numerically solve \autoref{eq:RRE} we need to discretize the spatial and reciprocal coordinates. We also normalize the parameters based on the magnitude of the incoming wave-vector $k$, so that $q$ and $p$ run over values close to unity. All parameters are rescaled in the following way:

\begin{align*}
    k \in (-\omega/c, \omega/c) \rightarrow|
\end{align*}

We know that $q = \frac{\omega}{v}$ and thus we can insert boundaries to the integral. We are dealing with EM-waves and thus set $Q$ to be 3-4 times larger than $\frac{\omega}{c}$ and the resulting integral will be:


The integral  we are to discretize and approximate this integral we can convert it into a sum:



where $q_n = nh_q, \; -\frac{N_q}{2} \leq n \leq \frac{N_q}{2}, \; h_q = \frac{Q}{N_q}$. We do the exact same to $p$ and choose $p_m = mh_p, \; -\frac{N_p}{2} \leq m \leq \frac{N_p}{2}, \; h_p = \frac{Q}{N_p}$ and choose $N_q = N_p$ for simplicity. Written like this we can again write this as a matrix equation on the similar form $Ax=b$:

\begin{equation}
    \frac{h_q}{2\pi}\boldsymbol{M}_\nu^+\boldsymbol{R}_{\nu} = \boldsymbol{M}_\nu^-,
\end{equation}

or written out, where the indices $\left(m, n \right)$ for $\left(p_m, q_n \right)$ are used:

% \hspace{-1cm} \begin{equation}
%     \frac{h_q}{2\pi}
%     \begin{bmatrix}
%     M^+_{\nu({-\frac{N_q}{2}}, {-\frac{N_q}{2}})} & M^+_{\nu({-\frac{N_q}{2}},{-\frac{N_q}{2}+1})} & . & . & . & M^+_{\nu({-\frac{N_q}{2}}, {\frac{N_q}{2}})}\\
%     M^+_{\nu({-\frac{N_q}{2}+1}, {-\frac{N_q}{2}})} & M^+_{\nu({-\frac{N_q}{2}+1},{-\frac{N_q}{2}+1})} & . & . & . & M^+_{\nu({-\frac{N_q}{2}+1}, {\frac{N_q}{2}})}\\
%     . & . & . &  &  & . \\
%     . & . &  & . &  & . \\
%     . & . &  &  & . & . \\
%     M^+_{\nu({\frac{N_q}{2}}, {-\frac{N_q}{2}})} & M^+_{\nu({\frac{N_q}{2}},{-\frac{N_q}{2}+1})} & . & . & . & M^+_{\nu({\frac{N_q}{2}}, {\frac{N_q}{2}})}\\
%     \end{bmatrix}
%     \begin{bmatrix}
%         R_{(-\frac{N_q}{2},k)}\\
%         R_{(-\frac{N_q}{2}+1,k)}\\
%         .\\
%         .\\
%         .\\
%         R_{(\frac{N_q}{2},k)}\\
%     \end{bmatrix}
%     =
%     \begin{bmatrix}
%         M^-_{\nu(-\frac{N_q}{2},k)}\\
%         M^-_{\nu(-\frac{N_q}{2}+1,k)}\\
%         .\\
%         .\\
%         .\\
%         M^-_{\nu(\frac{N_q}{2},k)}\\
%     \end{bmatrix}
% \label{R_Matrix_equation}
% \end{equation}

Notice here that $N_q$ is used for indexing both parameter $q$ and $p$. This is due to the earlier statement that $N_q = N_p$. To use this to find $\boldsymbol{R}_{\nu}$, it is necessary to find a approximation of the $I$ integral defined by Eq. (\ref{I_integral}). Where this will be evaluated in $\hat{Q} =p_m - q_n $, for a total of $2 N_q$ discrete values ranging between $-Q = -N_q h_q$ and $Q = N_q h_q$.
To find such an approximation, one can Taylor expand the expression the term $e^{i\gamma\zeta(x_1)}$, to only require the Fourier transform of the different orders of differentials of $\zeta(x_1)$.

\begin{equation}
    I(\gamma |\hat{Q}) = \sum^{\infty}_{n=0} \frac{(i\gamma)^n}{n!} \int dx_1 \zeta^{(n)}(x_1) exp(-i\hat{Q}x_1)
\end{equation}

%Vi kan slette likningen under her hvis vi får lite plass
Where we have used: 
\begin{equation}
    exp(i\gamma\zeta(x_1)) = \sum^{\infty}_{n=0}\frac{(i\gamma)^n \zeta^{(n)}(x_1)}{n!}
\end{equation}
%Her er slutten på hva vi kan sikkert slette hvis vi trenger 
This can approximated to order a Taylor expansion of order $N$, which gives an approximation defined by:

\begin{equation}
    I(\gamma|\hat{Q}) \approx I_N(\gamma|\hat{Q}) \equiv \sum^N_{n=0} \frac{(i\gamma)^n}{n!} \int dx_1 \zeta^{(n)}(x_1) exp(-i\hat{Q}x_1)
\end{equation}
In an implementation this can be solved for a finite length of $x_1$, which gives a finite length of values of $\zeta(x_1)$. This results in a discrete sampling of the Fourier transform, which is of interest to match with the chosen discretization of our wave numbers. The discretized wave numbers are separated by $h_q = \frac{Q}{N_q}$ with $p$ and $q$ ranging between $\{-\frac{Q}{2}, \frac{Q}{2} \}$, which results with the length being discretized with $N_x = 2 N_q $ points with $\zeta(x_1)$ ranging between $\{-\frac{L}{2}, \frac{L}{2} \}$ with separation of $\Delta = \frac{Q}{\pi}$ and length $L = \frac{2\pi N_q}{Q}$. The length of which we study is then determined by $Q$ and its number of discretizations $N_q$.

The result of the numerical approach is to fin $R_\nu(q|k)$ through solving Eq. (\ref{R_Matrix_equation}). This can then be compared with experiments by finding a physically measurable quantity which depends on this. One such physical quantity is the mean differential reflection coefficient, which describes the power of reflected light differentiated by the reflection angle. This is a function of the incoming waves angle $\theta_0$ \cite{Simonsen2010}. This is separated into the incoherent portion of the reflected light and the coherent giving two expressions given by:
\begin{equation}
    \left\langle \frac{\partial R_\nu}{\partial \theta_s}\right\rangle_{coh} = \frac{1}{L}\frac{\omega}{2\pi c}\frac{\cos^2 \theta_s}{\cos \theta_0} |\left\langle R_\nu(q|k) \right\rangle|^2
\label{DRC_coh}
\end{equation}
\begin{equation}
    \left\langle \frac{\partial R_\nu}{\partial \theta_s}\right\rangle_{incoh} = \frac{1}{L}\frac{\omega}{2\pi c}\frac{\cos^2 \theta_s}{\cos \theta_0} \left[ \left\langle |R_\nu (q|k) |^2 \right\rangle - |\left\langle R_\nu(q|k) \right\rangle|^2 \right]
\label{DRC_incoh}
\end{equation}
Where the angles $\theta_s$ and $\theta_0$ are given by the angle representations of the incoming wave by Eq. (\ref{k_by_angle}) and the reflected wave by Eq. (\ref{q_by_angle}). While $L$ is the length of our studied surface in the $x_1$ direction. 

Take $I_5(\gamma|Q), I_6(\gamma|Q), ... I_N(\gamma|Q)$, interpolation and let N -> $\infty$

Lenz summation to sum I

Start med diskretisering i $q$, så $p$ så $x_1$

Starte med en gaussian surface


Plot av absoluttkvadratet av den fouriertransformerte


\end{document}